\documentclass[a4paper,12pt]{article}
\usepackage[left=2.5cm,top=2.5cm,right=2.5cm,bottom=2.5cm]{geometry} 
\usepackage{selinput}
\usepackage{dirtytalk}
\usepackage[spanish]{babel}
\SelectInputMappings{
	aacute={á},
	ntilde={ñ},
	Euro={€}
}
\usepackage[T1]{fontenc}
\begin{document}
	\title{Breve historia de la computación en México durante 50 años \cite{texbook}}
	\author{Alyne Stephanie León Perez 5BM1}
	%\date{\today}
	\maketitle
	\font\logo=manfnt 
	\def\MF{{\logo META}\-{\logo FONT}}
	
	\noindent La primera computadora electrónica en México fue una IBM 650 instalada en la Universidad Nacional Autónoma de México (UNAM), en junio de 1958. El equipo responsable de este proyecto fue un grupo seleccionado de investigadores en Ingeniería, Física y Matemáticas que desempeñaban funciones clave en la UNAM, entre los que se encontraba Sergio Beltrán, quien fue nombrado primer director del Centro de Cómputo de la UNAM. En aquellos días, muy pocas personas sabían cómo manejar una computadora electrónica, solo unos pocos investigadores en física y astronomía tenían alguna experiencia en el uso de programas de computadora. Beltrán mostró un gran entusiasmo organizando el Centro de Cómputo con estudiantes de ingeniería y física, capacitándolos con el apoyo de IBM y organizando un total de cuatro coloquios anuales para investigadores y estudiantes con algunos de los mejores investigadores del momento, entre los que se encontraron Alan Perlis, McCarthy, Minsky, Niklaus Wirth y el {\bf Prof. Harold V. McIntosh}.\\
	De hecho, para hablar de la computación en México es preciso hablar del Prof. Harold V. McIntosh. En el 64 comenzó a trabajar en el ahora CINVESTAV del IPN; en el 65, en el CeNaC, fue miembro fundador de la Maestría en Computación. Posteriormente, continúa su investigación en Martin-Baltimore Computer Center, donde adquiere la experiencia en programación. Fue profesor en la Escuela de Física y Matemáticas, donde imparte cursos de la maestría en informática. Realizó investigaciones en el Centro de Cómputo Salazar del Instituto Nacional de Energía Nuclear, donde se desarrollaron programas y software en una red de computadoras. Después, en la Universidad Autónoma de Puebla ayudó a consolidar la Licenciatura en Informática y formar un proyecto de investigación informática con la intención de construir microcomputadoras. Escribió FLT y MBLISP y propuso el desarrollo de REC/Markov para los algoritmos de Markov.\\
	Durante los años 60, otras Universidades como el IPN y el Instituto Tecnológico de Monterrey (ITESM), instalaron sus propios centros de cómputo. Para 1968, regresa a México la primera generación de estudiantes mexicanos con un doctorado en Ciencias de la Computación, resultado de una política exitosa que envió estudiantes brillantes al extranjero con una beca para obtener su doctorado. A principios de los 80, había dos grupos principales de investigación con alrededor de 22 investigadores cada uno, uno en la UNAM y otro en la UAP. Desafortunadamente, debido a la crisis del 82, estos grupos se redujeron a 4 investigadores cada uno y el desarrollo científico y tecnológico dejó de ser una prioridad. Sin embargo, gracias al apoyo de las universidades y el excelente trabajo de los investigadores en informática, el campo creció sostenidamente desde mediados de la década de 1980. En 1986, se crea la Asociación Mexicana de Inteligencia Artificial (SMIA), en 1995 se crea la Asociación Mexicana de Ciencias de la Computación y desde entonces otras áreas habían lanzado sus propias asociaciones: Robótica, Interacción Hombre-Máquina, Procesamiento del Lenguaje Natural, etc. Este esfuerzo de agrupación había llevado a la creación en 2015 de la Academia Mexicana de Computación (AMEXCOMP), reconocida por el CONACyT.\\
	Durante los años 90 y 2000, podemos hablar de una segunda etapa de aportes inspirados y motivados por McIntosh. Uno de los temas más relevantes en los que McIntosh se dedicó en sus últimos años fue al problema de los autómatas celulares reversibles, incluyendo a estudiantes a través de los Veranos de Investigación. Actualmente, México cuenta con muy pocos grupos de investigación dedicados a la computación no convencional o natural. En el año 2018 se realizó la construcción de la primera máquina de Turing en México, la primera máquina de Turing robótica a nivel internacional. La máquina fue construida en el Laboratorio de Robótica de Vida Artificial (ALIROB) y el Laboratorio de Ciencias de la Computación de la Escuela Superior de Computo (ESCOM) del IPN.
	
	\begin{thebibliography}{9}
		\bibitem{texbook}
		Martínez G., Chapa S., Seck J., Lemaitre C. (2019). \emph{Brief Notes and History Computing in Mexico during 50 years}. 
	\end{thebibliography}
	
	
\end{document}