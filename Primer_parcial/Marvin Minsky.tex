\documentclass[a4paper,12pt]{article}
\usepackage[left=2.5cm,top=2.5cm,right=2.5cm,bottom=2.5cm]{geometry} 
\usepackage{selinput}
\usepackage{dirtytalk}
\usepackage[spanish]{babel}
\SelectInputMappings{
  aacute={á},
  ntilde={ñ},
  Euro={€}
}
\usepackage[T1]{fontenc}
\begin{document}
\title{Biografía de Marvin Minsky \cite{texbook}}
\author{Alyne Stephanie León Perez 5BM1}
\date{\today}
\maketitle
\font\logo=manfnt 
\def\MF{{\logo META}\-{\logo FONT}}

\noindent{\bf Marvin Minsky} fue profesor \say{Toshiba} de Artes y Ciencias de Medios, y profesor de Ingeniería Eléctrica y Ciencias de la Computación en el Instituto Tecnológico de Massachusetts. Su investigación condujo a avances teóricos y prácticos en el área de Inteligencia Artificial, ciencia cognitiva, redes neuronales y teoría de las máquinas de Turing y funciones recursivas, además de otras contribuciones en los dominios de gráficos, computación matemática simbólica, representación del conocimiento, semántica del sentido común, y aprendizaje simbólico y coneccionista.\\
El profesor Minsky fue un pionero en robótica. Él diseñó y construyó brazos mecánicos, manos con sensores táctiles, y una de las primeras tortugas del lenguaje \say{Logo}. Ésto influenció muchos proyectos robóticos subsecuentes. También estuvo involucrado con tecnologías avanzadas para la exploración espacial y fue un consultor en la película \say{Odisea del espacio} de Stanley Kubrick.\\
En 1951, construyó la primera red neuronal aleatoriamente conectada llamada SNARC (Computadora de Refuerzo Neuro-Análogo Estocástico, por sus siglas en inglés), basado en conexiones de refuerzo sinápticas. En 1956, siendo un \textit{Junior Fellow} en Harvard, inventó y construyó el primer Microscopio de Barrido Confocal, un instrumento óptico con una resolución y calidad de imagen sin precedentes. Después de principios de la década de 1950, Minsky trabajó en utilizar ideas computacionales para caracterizar procesos psicológicos humanos, y también en dotar máquinas con inteligencia.\\
A principios de 1970, él y Seymour Papert empezaron a formular una teoría llamada \textit{Sociedad de la Mente} que combinaba ideas de psicología del desarrollo infantil y su experiencia con el desarrollo en inteligencia artificial. La \textit{Sociedad de la Mente} propone que la inteligencia no es el producto de ningún mecanismo singular, sino, viene de interacciones administradas de una variedad de agentes.\\
En 1985, Minsky publicó \textit{La Sociedad de la Mente}, un libro en el cual ideas de \say{una página} interconectadas reflejan la estructura de la teoría en sí misma.\\
En 2006, publicó una secuela: \textit{La máquina de emociones}, la cual propone teorías que podrían explicar los sentimientos humanos, metas, emociones y pensamientos conscientes de alto nivel en términos de procesos de múltiples niveles, algunos de los cuales se pueden reflejar en los demás. Al proporcionarnos múltiples y diferentes \say{maneras de pensar}, estos procesos podrían explicar gran parte de nuestro ingenio humano único.\\
Además de sus logros técnicos, Marvin fue un consumado pianista clásico. se enseñó a sí mismo a improvisar en el estilo de Bach y Beethoven, y continuó tocando hasta su muerte, el 24 de enero del 2016.
\begin{thebibliography}{9}
\bibitem{texbook}
Solomon, C. (2019). \emph{Inventive minds: Marvin Minsky on Education}, MIT Press, London, England.
\end{thebibliography}


\end{document}