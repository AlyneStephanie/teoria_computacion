\documentclass[a4paper,12pt]{article}
\usepackage[left=2.5cm,top=2.5cm,right=2.5cm,bottom=2.5cm]{geometry} 
\usepackage{selinput}
\usepackage{dirtytalk}
\usepackage[spanish]{babel}
\SelectInputMappings{
	aacute={á},
	ntilde={ñ},
	Euro={€}
}
\usepackage[T1]{fontenc}
\begin{document}
	\title{Las matemáticas y los problemas computables \cite{texbook}}
	\author{Alyne Stephanie León Perez 5BM1}
	\date{\today}
	\maketitle
	\font\logo=manfnt 
	\def\MF{{\logo META}\-{\logo FONT}}
	
	\noindent La formación de un estudiante debe ser integral. Primero que nada, debe tenerse un fondo histórico; el razonamiento matemático debe ser innegable. Otra cuestión es saber expresarse correctamente; la investigación debe ser fundamental: la educación no avanza si no hay investigación. Otro punto importante en la formación del estudiante es la programación. No importa si se trata de un campo físico matemático, social o biológico, debe haber un punto en el que debe orientarse hacia la programación.
	La importancia de las matemáticas es impresionante. Remontándonos a 1900, ocurrió en París el \textit{\say{Congreso Internacional de Matemáticas}}. Durante esta conferencia se plantearon diversos problemas. Dentro de estos problemas, concentrándonos particularmente en el problema de Hilbert, éste planteaba cómo poder encontrar un método para resolver ecuaciones diofánticas.\\ 
	En aquel entonces, varios matemáticos ya venían desarrollando una teoría, la cual trataba de explicar por qué la lógica matemática podía resolverse de manera sistemática. Le tomó 10 años a Martin Davis encontrar la solución del problema, 70 años después del Congreso. La solución es que no tiene solución.\\
	Alan Turing desarrolló un método, el procedimiento efectivo, basado en una estructura algebráica. La característica principal de este método era su simpleza. Turing encuentra una correspondencia entre sus máquinas y los problemas planteados en matemáticas, y plantea un método de sistemas universales; es decir, a través de elementos básicos se puede construir todo el universo de \textit{ese universo}.
	La peculiaridad de la máquina universal que propone Turing es que ésta es capaz de reconocer el tipo de problema que se tiene y emplear la máquina adecuada. Ésto no significa que se pueda resolver cualquier problema que se pueda presentar en un programa computable. Un problema de esta máquina universal es encontrar la máquina universal más pequeña que tenga el orden de complejidad más pequeño.\\
	¿Qué es computación? Todo sistema debe tener una condición inicial, condiciones de entrada, una descripción del problema y, entonces, conocer cada una de las instancias del problema. Una de las problemáticas es que puede tener un conjunto de soluciones, y dichas soluciones pueden ser finitas o infinitas. Todo esto nos lleva a los problemas computables y no computables; si un problema es computable, tenemos una máquina de Turing asociada a ese problema.
	
	
	
	\begin{thebibliography}{9}
		\bibitem{texbook}
		Juárez, Genaro (2021). \emph{Las matemáticas y los problemas computables}, ESCOM, Ciudad de México. https://www.comunidad.escom.ipn.mx/genaro/
	\end{thebibliography}
	
	
\end{document}